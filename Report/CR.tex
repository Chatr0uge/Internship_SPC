\documentclass[11pt,a4paper]{report}
\usepackage[utf8]{inputenc}
\usepackage[english]{babel}
\usepackage[T1]{fontenc}
\usepackage{amsmath}
\usepackage{amsfonts}
\usepackage{amssymb}
\usepackage{makeidx}
\usepackage{graphicx}
\usepackage{float}
\usepackage{lmodern}
\usepackage[dvipsnames]{xcolor}
\usepackage{tikz}
\usetikzlibrary{intersections}
\usepackage{pgfplots}
\usetikzlibrary{calc}
\usepackage{geometry}
\geometry{hmargin=2cm,vmargin=2cm}
\usepackage{fancybox}
\usepackage{mathtools}
\usepackage{enumitem}
\usepackage{tcolorbox}
\usepackage{colortbl}
\usepackage{fancybox}
\tcbuselibrary{most}
\usepackage{pifont}
\usepackage[skip = 2pt, font = footnotesize]{caption}
\usepackage{subcaption}
\usepackage{eso-pic}
\usepackage{nicematrix}
\usepackage{multicol}
\usepackage{booktabs}
\usepackage{svg}
\usepackage{derivative}
\usepackage{wrapfig}
\usepackage{stmaryrd}
\usepackage{yfonts}
\usepackage[backend=biber]{biblatex}
\addbibresource{references.bib}
\author{Andrea}
\setlength{\columnsep}{.5cm}
\renewcommand{\thesection}{\Roman{section}}
\renewcommand{\thesubsection}{\arabic{subsection}}
\renewcommand{\thesubsubsection}{\alph{subsubsection}}
\usepackage{amsmath}
\colorlet{shadecolor}{cyan!15}
\usepackage{fancyhdr}
\usepackage{etoolbox}
\usepackage[export]{adjustbox}
\usepackage{fourier-orns}
\usepackage{lettrine}
\usepackage{physics}
\definecolor{RoyalRed}{RGB}{157,16, 45}
\usepackage{titlesec}
\usepackage{lipsum} 
\titleclass{\chapter}{straight}
\titleformat{\chapter}[display]
{\normalfont\bfseries\filcenter}
{\color{black}\LARGE\thechapter}
{1ex}
{\color{black}\titlerule[2pt]
\vspace{2ex}%
\LARGE}
[\vspace{1ex}%
{\titlerule[2pt]}]

\usepackage[export]{adjustbox}
\renewcommand{\headrule}{%
\vspace{6pt}\hrulefill
\raisebox{0pt}{\quad\decofourleft\decotwo\decofourright\quad}\hrulefill}
\pagestyle{fancy}
\fancyhf{}
\rhead{ \textcolor{black}{\footnotesize \today}}
\lhead{ENS}
\chead{ \textcolor{black}{· \emph{Turbulence characterization in tokamak} ·}}
\rfoot{Andrea Combette}
\fancyfoot[C]{\thepage} 

\newlength{\tabcont}

\setlength{\parindent}{0.0in}
\setlength{\parskip}{0.05in}

\setcounter{tocdepth}{4}
\setcounter{secnumdepth}{4}

\begin{document}
\begin{titlepage}
    \AddToShipoutPictureBG*{
        \begin{tikzpicture}[overlay,remember picture]
            \draw [line width=3pt]
            ($ (current page.north west) + (2cm,-2.0cm) $)
            rectangle
            ($ (current page.south east) + (-2cm,1.8cm) $);
            \draw [line width=1pt]
            ($ (current page.north west) + (2.15cm,-2.15cm) $)
            rectangle
            ($ (current page.south east) + (-2.15cm,1.95cm) $);
        \end{tikzpicture}
    }
    \begin{center}
        \vspace*{2cm}
        \emph{\footnotesize{Department of physics, École Normale Supérieure, Paris}}

        \emph{\footnotesize{Swiss Plasma Center, EPFL, Lausanne}}


        \vspace*{1cm}

        \textsc{Turbulence characterization}

        \textsc{In magnetically confined fusion research}
        \vspace*{1cm}

        \rule{14cm}{2pt}\vspace{.7cm}

        \Large{\textbf{Master Thesis 2024}}

        \vspace{.5cm}
        \rule{14cm}{2pt}
        \vspace{1cm}

        \Large Andrea Combette

        \vspace{3cm}

        \raisebox{-5pt}{\quad\decofourleft\decotwo\decofourright\quad}

        \vspace{2cm}
        \vspace{1cm}

        \begin{minipage}{14cm}
            \small{\textbf{Supervisors:}}
            \vspace{.5cm}

            \small{\textbf{Mr. Oleg Krutkin \null\hfill Pr. Jean François Allemand}}

        \end{minipage}
        \vspace{2cm}


        \begin{minipage}{14cm}
            \small{
                \textbf{Cautionary note : } This paper is a report on numerical methods for the shallow water equations and gravity waves. It is not intended to be a complete and rigorous study of the subject. The reader is invited to refer to the references for further details.
                It has been made by a Master Student, with some background in physics and mathematics, but no prior knowledge of the subject. It is therefore not intended to be a reference for experts in the field.}
        \end{minipage}

    \end{center}

\end{titlepage}

\newpage
\tableofcontents
\newpage


\begin{center}
    \vspace*{1.5cm}\Large{\textbf{Abstract}}
    \vspace*{1cm}
    \fontsize{11}{18}\selectfont

    \begin{minipage}{.7\linewidth}
        \lettrine[lines=4]{\color{black} O}{ne} of the common goals in experimental magnetically confined fusion research is characterization of the plasma turbulence. To that end, TCV tokamak features a novel short-pulse reflectometry (SPR) diagnostic, which can potentially be utilized to measure properties of the turbulence.
        It is essentially a radar system, where the plasma is probed by a short (under ns) microwave pulse in the presence of the cut-off (reflection) area from which the pulse reflects back into the probing antenna. The position of the cut-off for a particular probing frequency (in 50-75 GHz) range is determined by the plasma electron density. Thus, by measuring the delay between probing and reflected beam corresponding to different probing frequencies, the information about the electron density profile is inferred including its turbulent perturbations.
        Unfortunately, the complex interaction of microwaves with magnetized plasma makes it difficult to establish the connection between SPR measurements and properties of the turbulence. Numerical modeling utilizing the synthetic diagnostic approach was carried out to establish this connection for the case of low turbulence amplitudes (linear regime). However, the case of large turbulence amplitudes (nonlinear regime) is yet to be explored.
        Within the project a systematic analysis of the SPR diagnostic in the nonlinear regime will be carried out. The numerical finite difference code CUWA, which solves the wave equation for a given plasma density and provides synthetic reflected pulse will be utilized. The main goal of the project is identifying markers that can be used to determine if the diagnostic is operating in the nonlinear regime and assessing the possibility of determining the turbulence parameters regardless. Time permitting, the results of this analysis will be applied to the interpretation of experimental measurements and possibly used to develop a machine learning approach to analyzing SPR data.
    \end{minipage}
\end{center}

\newpage
\fontsize{10}{10}\selectfont
\begin{multicols}{2}
    \chapter{Theoretical Background}
    \section{Nuclear Fusion}
    \lettrine[lines=2, lhang=.3, nindent=0pt]{\color{black} T}{he} sh

    \subsection{Reaction}
    \subsubsection{D-T reaction}
    \subsubsection{Cross-section}
    $\sigma \approx \sigma _{\text{geometry}}\times T\times R,$
    \subsubsection{Energy balance and Lawson criterion}
    $ \tau _{E}={\frac {W}{P_{\mathrm {loss} }}}$

    For the deuterium–tritium reaction, the physical value is at least

    $$n \tau E \ge 1.5.10^{20}{\frac {\mathrm {s} }{\mathrm {m} ^{3}}}$$
    Different regimes of confinement : Magnetic, Inertial, \dots
    \subsection{Tokamak}
    Magnetic confinement, plasma physics, \dots

    \section{{Plasma Turbulence}}
    \subsection{Characterization}
    instabilities grow due to inverse cascade of energy (2D geometry) --> scale
    \subsection{Diagnostics}
    \subsubsection{Doppler Reflectometry}
    Doppler reflectometry is a radar technique that measures the density fluctuations in a plasma. It is based on the interaction between probing microwave  and the turbulence, indeed the back-scattered wave by the plasma, is amplitude and phase dependent of many turbulence parameters. The Doppler shift is proportional to the velocity of the plasma fluctuations along the direction of the probing wave. The technique is used to measure the density fluctuations in the edge of the plasma, where the turbulence is strongest. The technique is also used to measure the radial electric field in the edge of the plasma, which is important for understanding the transport of particles and heat in the plasma. Doppler reflectometry is used in the TCV tokamak to measure the turbulence parameters (amplitude, mode number ... ).

    \subsubsection{RCDR}
    RCDR is a advanced Doppler Reflectometry technique it allows to measure the radial correlation length of the turbulences, using CCF. The principle is the following, we vary the probing wave frequency and we measure the correlation of the back-scattered wave. The correlation length is then deduced from the CCF function analysis.

    \subsubsection{Short Pulse Reflectometry}
    SPR is a radar technique that measures the density fluctuations in a plasma. It is based on the interaction between probing microwave  and the turbulence, indeed the back-scattered wave by the plasma, is amplitude and phase dependent of many turbulence parameters. The technique is used to measure the density fluctuations in the edge of the plasma, where the turbulence is strongest. The technique is also used to measure the radial electric field in the edge of the plasma, which is important for understanding the transport of particles and heat in the plasma. SPR is used in the TCV tokamak to measure the turbulence parameters (amplitude, mode number ... ).
    \chapter{Numerical Modeling}
    \section{Numerical Integration for plasma density}
    \section{Full wave Modelling}



\end{multicols}
\printbibliography
\end{document}