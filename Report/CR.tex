\documentclass[11pt,a4paper]{report}
\usepackage[utf8]{inputenc}
\usepackage[english]{babel}
\usepackage[T1]{fontenc}
\usepackage{amsmath}
\usepackage{amsfonts}
\usepackage{amssymb}
\usepackage{makeidx}
\usepackage{graphicx}
\usepackage{float}
\usepackage{lmodern}
\usepackage[dvipsnames]{xcolor}
\usepackage{tikz}
\usetikzlibrary{intersections}
\usepackage{pgfplots}
\usetikzlibrary{calc}
\usepackage{geometry}
\geometry{hmargin=2cm,vmargin=2cm}
\usepackage{fancybox}
\usepackage{mathtools}
\usepackage{enumitem}
\usepackage{tcolorbox}
\usepackage{colortbl}
\usepackage{fancybox}
\tcbuselibrary{most}
\usepackage{pifont}
\usepackage[skip = 2pt, font = footnotesize]{caption}
\usepackage{subcaption}
\usepackage{eso-pic}
\usepackage{nicematrix}
\usepackage{multicol}
\usepackage{booktabs}
\usepackage{svg}
\usepackage{derivative}
\usepackage{wrapfig}
\usepackage{stmaryrd}
\usepackage{yfonts}
\usepackage[backend=biber]{biblatex}
\addbibresource{references.bib}
\author{Andrea}
\setlength{\columnsep}{.5cm}
\renewcommand{\thesection}{\Roman{section}}
\renewcommand{\thesubsection}{\arabic{subsection}}
\renewcommand{\thesubsubsection}{\alph{subsubsection}}
\usepackage{amsmath}
\colorlet{shadecolor}{cyan!15}
\usepackage{fancyhdr}
\usepackage{etoolbox}
\usepackage[export]{adjustbox}
\usepackage{fourier-orns}
\usepackage{lettrine}
\usepackage{physics}
\definecolor{RoyalRed}{RGB}{157,16, 45}
\usepackage{titlesec}
\usepackage{lipsum} 
\titleclass{\chapter}{straight}
\titleformat{\chapter}[display]
{\normalfont\bfseries\filcenter}
{\color{black}\LARGE\thechapter}
{1ex}
{\color{black}\titlerule[2pt]
\vspace{2ex}%
\LARGE}
[\vspace{1ex}%
{\titlerule[2pt]}]

\usepackage[export]{adjustbox}
\renewcommand{\headrule}{%
\vspace{6pt}\hrulefill
\raisebox{0pt}{\quad\decofourleft\decotwo\decofourright\quad}\hrulefill}
\pagestyle{fancy}
\fancyhf{}
\rhead{ \textcolor{black}{\footnotesize \today}}
\lhead{ENS}
\chead{ \textcolor{black}{· \emph{Turbulence characterization in tokamak} ·}}
\rfoot{Andrea Combette}
\fancyfoot[C]{\thepage} 

\newlength{\tabcont}

\setlength{\parindent}{0.0in}
\setlength{\parskip}{0.05in}

\setcounter{tocdepth}{4}
\setcounter{secnumdepth}{4}

\begin{document}
\begin{titlepage}
    \AddToShipoutPictureBG*{
        \begin{tikzpicture}[overlay,remember picture]
            \draw [line width=3pt]
            ($ (current page.north west) + (2cm,-2.0cm) $)
            rectangle
            ($ (current page.south east) + (-2cm,1.8cm) $);
            \draw [line width=1pt]
            ($ (current page.north west) + (2.15cm,-2.15cm) $)
            rectangle
            ($ (current page.south east) + (-2.15cm,1.95cm) $);
        \end{tikzpicture}
    }
    \begin{center}
        \vspace*{2cm}
        \emph{\footnotesize{Department of physics, École Normale Supérieure, Paris}}

        \emph{\footnotesize{Swiss Plasma Center, EPFL, Lausanne}}


        \vspace*{1cm}

        \textsc{Turbulence characterization}

        \textsc{In magnetically confined fusion research}
        \vspace*{1cm}

        \rule{14cm}{2pt}\vspace{.7cm}

        \Large{\textbf{Master Thesis 2024}}

        \vspace{.5cm}
        \rule{14cm}{2pt}
        \vspace{1cm}

        \Large Andrea Combette

        \vspace{3cm}

        \raisebox{-5pt}{\quad\decofourleft\decotwo\decofourright\quad}

        \vspace{2cm}
        \vspace{1cm}

        \begin{minipage}{14cm}
            \small{\textbf{Supervisors:}}
            \vspace{.5cm}

            \small{\textbf{Dr Mr. Oleg Krutkin \null\hfill Pr. Jean François Allemand}}

        \end{minipage}
        \vspace{2cm}


        \begin{minipage}{14cm}
            \small{
                \textbf{Cautionary note : } This paper is a report on numerical methods for the shallow water equations and gravity waves. It is not intended to be a complete and rigorous study of the subject. The reader is invited to refer to the references for further details.
                It has been made by a Master Student, with some background in physics and mathematics, but no prior knowledge of the subject. It is therefore not intended to be a reference for experts in the field.}
        \end{minipage}

    \end{center}

\end{titlepage}

\newpage
\tableofcontents
\newpage


\begin{center}
    \vspace*{1.5cm}\Large{\textbf{Abstract}}
    \vspace*{1cm}
    \fontsize{11}{18}\selectfont

    \begin{minipage}{.7\linewidth}
        \lettrine[lines=4]{\color{black} O}{ne} of the common goals in experimental magnetically confined fusion research is characterization of the plasma turbulence. To that end, TCV tokamak features a novel short-pulse reflectometry (SPR) diagnostic, which can potentially be utilized to measure properties of the turbulence.
        It is essentially a radar system, where the plasma is probed by a short (under ns) microwave pulse in the presence of the cut-off (reflection) area from which the pulse reflects back into the probing antenna. The position of the cut-off for a particular probing frequency (in 50-75 GHz) range is determined by the plasma electron density. Thus, by measuring the delay between probing and reflected beam corresponding to different probing frequencies, the information about the electron density profile is inferred including its turbulent perturbations.
        Unfortunately, the complex interaction of microwaves with magnetized plasma makes it difficult to establish the connection between SPR measurements and properties of the turbulence. Numerical modeling utilizing the synthetic diagnostic approach was carried out to establish this connection for the case of low turbulence amplitudes (linear regime). However, the case of large turbulence amplitudes (nonlinear regime) is yet to be explored.
        Within the project a systematic analysis of the SPR diagnostic in the nonlinear regime will be carried out. The numerical finite difference code CUWA, which solves the wave equation for a given plasma density and provides synthetic reflected pulse will be utilized. The main goal of the project is identifying markers that can be used to determine if the diagnostic is operating in the nonlinear regime and assessing the possibility of determining the turbulence parameters regardless. Time permitting, the results of this analysis will be applied to the interpretation of experimental measurements and possibly used to develop a machine learning approach to analyzing SPR data.
    \end{minipage}
\end{center}

\newpage
\fontsize{10}{10}\selectfont
\begin{multicols}{2}
    \chapter{Theoretical Background}
    \section{Nuclear Fusion}
    \lettrine[lines=2, lhang=.3, nindent=0pt]{\color{black} T}{he} nuclear fusion reaction is the process by which two light atomic nuclei combine to form a heavier nucleus. It is accompanied by the release or absorption of energy depending on the masses of the nuclei involved. Indeed, the more the nuclei are light, the  more  energy is released due to the overcoming short-range nuclear force for light nuclei.
    However, to overcome the Coulomb barrier the reactant must be sufficiently close for a long enough time to allow the quantum tunnel effect between both particles. To do so, we must heat up the reactant to huge temperatures such that these latter are starting ionizing and turning into plasma.

    \subsection{Reaction}
    \begin{figure}[H]
        \includegraphics[width=1\linewidth]{./figures/potential_barrier2.pdf}
        \caption{Here we plot the residual Coulomb potential barrier between two proton considering the strong nuclear force as the Reid potential [cite]. On the right, the residual Reid potential.}
        \label{fig:barrier}
    \end{figure}
    Here it appears that the thermal energy needed to overcome the coulomb barrier is about : $$E_{\text{thermal}} \approx 0.2 \text{MeV}.$$This corresponds to a temperature of $2.5e^9$ K,  This is the reason why we need to heat up the plasma to such high temperatures. Note that the quick considerations are for proton-proton interaction, in fact the thermal energy needed is much smaller around $1e^8$ K for the deuterium-tritium reaction, due to quantum tunneling effect, and screening of coulomb potential by other nucleons:

    \subsubsection{D-T reaction}
    At the TCV the studied reaction is the deuterium-tritium reaction, which is the most promising reaction for fusion power, Indeed the energy barrier for the reaction to happen is about $70$ keV, whereas for deuterium-deuterium reaction it's $0.$1MeV,and for the deuterium-Helium it's about $0.2$MeV. The reaction is the following:
    \begin{align*}
        \text{D}^2 + \text{T}^3 & \rightarrow \text{He}^4 + \text{n(17.6 MeV)}
    \end{align*}
    The liberated energy is big enough to sustain the reaction, but it has a given probability to happen, and a positive energy yield is necessary to use the nuclear fusion reaction.
    A possible measure of this probability is the fusion cross-sectional area, which is much more that just a geometrical cross-section.
    \subsubsection{Fusion Cross-section}
    The fusion cross section is enhanced by the tunneling effect transparency ($T$) and by the reaction characteristics $R$. It is given by the following formula :
    $$\sigma \approx \sigma _{\text{geometry}}\times T\times R,$$

    where $\sigma _{\text{geometry}}$ is the geometrical cross-section, $T$ is the tunneling effect transparency, and $R$ is the reaction characteristics. The tunneling effect transparency is given by the Gamow factor, and the reaction characteristics given by the astrophysical S-factor.
    \subsubsection{Energy balance and Lawson criterion}
    $ \tau _{E}={\frac {W}{P_{\mathrm {loss} }}}$

    For the deuterium–tritium reaction, the physical value is at least

    $$n \tau E \ge 1.5.10^{20}{\frac {\mathrm {s} }{\mathrm {m} ^{3}}}$$
    Different regimes of confinement : Magnetic, Inertial, \dots
    \subsection{Tokamak}
    Magnetic confinement, plasma physics, \dots
    \section{Wave propagation in plasma}
    \subsection{Plasma as a medium}
    \subsection{Wave equation}
    \subsection{Dispersion relation}
    \subsubsection{Ordinary mode}
    \subsubsection{Extraordinary mode}

    \section{{Plasma Turbulence}}
    \subsection{Characterization}
    instabilities grow due to inverse cascade of energy (2D geometry) --> scale
    \subsection{Diagnostics}
    \subsubsection{Doppler Reflectometry}

    \subsubsection{RCDR}

    \subsubsection{Short Pulse Reflectometry}

    \chapter{Numerical Modeling}

    \section{Numerical Integration for plasma density}
    Kinetic model for plasma, equations \dots
    \section{Full wave Modelling}
    CUWA CODE : Finite difference method, wave equation, plasma density, \dots

    \chapter{Simulations Results}
    \section{Linear Regime Study}
    \section{1 dimensional study}
    Assuming a simple plasma density profile $n(x)$, we can study the wave propagation in the plasma. The goal of this approach is to find a way to link the plasma density perturbations to the reflected pulse delay. To retrieve some information about the pulse delay we will
    use a statistical approach to get rid of the randomness implies by the perturbations considerations.
    The delay of the probing wave is given by the following formula  : $$\tau_c = 2 \int_0^L \frac{dx}{v_g}$$ Where $v_g$ is the group velocity of the wave, $L$ is the position of the cut-off.
    From the simple assumption $\langle \delta n \rangle = 0 $ for an Ordinary mode the  $v_g$ expression obtained [] can be used to expand the integral to the following :
    $$\frac{2}{c} \int_0^L \frac{dx}{\sqrt{1 - \frac{x}{L} - \frac{\delta n }{n_c}}}$$ The main contribution of this integral comes from the vicinity of the cut-off layer, Where the group velocity is the smallest.
    We can discuss the relevance of this expansion this the main contribution of the integral comes from the cut-off region where the WKB approximation cannot be applied.

    \subsection{Perturbation Density Profile}
    \subsubsection{Step-like perturbation}
    The apparent divergence of the integral can be tackled using a simple step-like perturbation density profile.
    With a step-size perturbation characterized by $l_{cx}$ length. This allows to get an analytical expression of the integral for different density profile. However, to get this simplification, we need to assume that the perturbation is small enough such that the WKB approximation can be applied.
    This is the case for the linear regime, where the perturbation is small enough such that the cut-off layer is not too much perturbed (i.e $\delta_x \ll l_{cx}$). In the case of a large perturbation, an other step perturbation localized far from the cut-off layer can be used to get the same result, which breaks the main assumption of this approach.
    It's relatively trivial to obtain the following expression for the delay [cite Krutkin] :
    $$\tau_d = \frac{4L}{c} - \frac{2L}{c}\sqrt{\frac{L}{l_{cx}}}\frac{\delta n}{n_c} $$
    \subsubsection{Gaussian-like perturbation}
    The gaussian perturbation should be a more realistic approach to the perturbation density profile. However, the integral seems to be more difficult to solve. Indeed, it takes the following form :

    $$\frac{2}{c} \int_0^L \frac{dx}{\sqrt{1 - \frac{x}{L} - \frac{\delta n\exp(-\frac{(x - L)^2}{8l_{cx}^2})}{n_c}}}$$
    First let's note that the main criterion $\delta n_c = n_c \frac{l_{cx}}{L}$ is the same as the previous approach.  One way to tackle this integral is to expand the gaussian perturbation profile to the first order in order to obtain a hyperbolic integral. The computation are done in the appendix, and leads to the following expression of the delay :

\end{multicols}

\begin{figure}[H]
    \centering
    \includegraphics[width = 1\linewidth]{./figures/density_profile.png}
    \caption{sadasd}
    \label{}
\end{figure}


\begin{figure}[H]
    \centering
    \includegraphics[width=1\linewidth]{./figures/delay_amp_norm_2.png}
    \caption{Here we plot the residual Coulomb potential barrier between two proton considering the strong nuclear force as the Reid potential [cite]. On the right, the residual Reid potential.}
    \label{fig:barrier}
\end{figure}
\begin{multicols}{2}



    \section{2 dimensional study}

    \section{Nonlinear Regime Study}

\end{multicols}
\printbibliography
\end{document}